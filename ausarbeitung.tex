% This is LLNCS.DEM the demonstration file of
% the LaTeX macro package from Springer-Verlag
\documentclass[a4paper,12pt]{llncs}
%
\usepackage{makeidx}  % allows for indexgeneration
\makeindex

\usepackage[ngerman]{babel}
\usepackage[utf8]{inputenc}      % Code-Page latin 1
\usepackage[T1]{fontenc}
% Nur eine der beiden folgenden Zeilen einbinden!
% siehe Abschnitt Bilder
%\usepackage{graphicx}       % Bilder einbinden, Version fuer normales latex
\usepackage[pdftex]{graphicx}       % Bilder einbinden, Version fuer pdflatex

% mit Hyperrefs
\usepackage[pdftex, plainpages=false,hypertexnames=true,pdfnewwindow=true,backref=true,colorlinks=true,citecolor=blue,linkcolor=black,urlcolor=blue,filecolor=blue]{hyperref}% 
% weitere Packages
\usepackage{ifthen}                 % Zum Auskommentieren von Textteilen
\usepackage{amssymb}                % Mathematische Buchstaben
\usepackage{amsmath}                % Verbesserter Formelsatz
\usepackage[vlined,boxed]{algorithm2e}
\usepackage{booktabs}               % schönere Tabellen
\usepackage{color}
\usepackage{hyperref}
 \hypersetup{urlcolor=black,citecolor=black}
%\setalcapskip{1.5ex} % fuer package algorithm
\usepackage{dsfont}  
%\newtheorem{definition}{Definition}
\usepackage{doc}

\usepackage{subcaption}

% Seitenformat ===============================================================
\hoffset=-1.25truecm
\setlength{\topmargin}{0.0cm}
\setlength{\textheight}{23.0cm}
\setlength{\footskip}{1.5cm}
\setlength{\textwidth}{15.4cm}
\setlength{\evensidemargin}{1.5cm}
\setlength{\oddsidemargin}{1.5cm}
\setlength{\parskip}{1ex}
\setlength{\parindent}{0pt}
\setlength{\marginparwidth}{1.4cm}
\setlength{\marginparsep}{1mm}
\setalcapskip{1.5ex} % fuer package algorithm

\pagestyle{plain}

% Makro-Definitionen ==========================================================
% Zahlenbereiche -------------------------------------------------------------
\newcommand{\N}{{\mathbb{N}}}
\newcommand{\R}{{\mathbb{R}}}
\newcommand{\C}{{\mathbb{C}}}
\newcommand{\Z}{{\mathbb{Z}}}
\newcommand{\Q}{{\mathbb{Q}}}

% 
\def\myverzeichnis{.}

\numberwithin{equation}{section} 
% Bild -----------------------------------------------------------------------
% #1 Filename;  #2 Label;  #3 Bildunterschrift;  #4 Kurzform
\newcommand{\bild}[4]{
  \begin{figure}[htbp]
    \begin{center}
      \includegraphics{#1}
      \caption[#4]{#3}
      \label{#2}
    \end{center}
  \end{figure}
}

% Bildbreite -----------------------------------------------------------------
% #1 Filename;  #2 Breite;  #3 Label;  #4 Bildunterschrift;  #5 Kurzform
\newcommand{\bildbreite}[5]{
  \begin{figure}[htbp]
    \begin{center}
      \includegraphics[width=#2]{#1}
      \caption[#5]{#4}
      \label{#3}
    \end{center}
  \end{figure}
}

% !TeX spellcheck = de_DE
% ============================================================================
\begin{document}

% =========== Das war der Vorspann, jetzt geht's los! ========================

% ============================================================================
% =============  AB HIER DARF UND SOLL GETIPPT WERDEN ========================
% ============================================================================

\author{Yaroslav Nalivayko}
\index{Yaroslav Nalivayko}

% Das Institut wird fuer den Betreuer missbraucht ...
\institute{{\bf Betreuer:} M.Sc. Benjamin Maier}
\authorrunning{Yaroslav Nalivayko}
\title{Newton Fraktale}

\maketitle

\thispagestyle{empty}

\begin{abstract}
Newton Fraktale sind eine Teilmenge der mathematischen Fraktale, die durch Benutzung des Newton Verfahrens für Lösung von nichtlinearen Gleichungen auf Komplexe Ebene erscheinen.
\end{abstract}

% Einleitung -----------------------------------------------------------------
\section{Einleitung}
Newton Fraktale stellen eine interessante Klasse der mathematischen Fraktalen dar. 
Im Rahmen dieser Arbeit werden \nameref{sec:theo} vorgestellt und ein Programm für die \nameref{sec:vis} der Fraktalen entwickelt. In letzte Sektion findet \nameref{sec:analy} mancher interessanten Funktionen statt.


\section{Theoretische Grundlagen}\label{sec:theo}
%Hier werden Grundbegriffe erläutert.
\subsection{Numerische Mathematik}
Die numerische Mathematik beschäftigt sich als Teilgebiet der Mathematik mit der Konstruktion und Analyse von Algorithmen für kontinuierliche mathematische Probleme. \cite{nummath}
Oft wird benutzt, um approximative Lösungen mit Hilfe von Computer zu finden.
\subsection{Newton Verfahren}
Newton Verfahren ist das iterative numerische Verfahren, das eine Wurzel gegebener Funktion findet.	Die Methode ist nach Sir Isaac Newton benannt. \\
Wir interessieren uns in stetig differenzierbaren Funktionen mit nur eine Variable.
\[
f(x) = 0
\] 
Man soll manuell den Startwert $x_0$ wählen und dann die iterative Methode benutzen, bis akzeptable Lösung gefunden wird.
\[
x_{n+1} = x_n + \frac{f(x_n)}{f'(x_n)}
\] 
Gewöhnlich wählt man eine zulässige Abweichung $\varepsilon$ und eine maximale Anzahl der Schritte $N$.
Nach jedem Schritt der iterative Methode prüft man. Falls $f(x_n)  < \varepsilon$, dann ist die Lösung gefunden. Und falls $n > N$, dann ist die Lösung unauffindbar in akzeptable Anzahl der Schritte. 
\subsection{Fraktale}
Fraktal (lateinisch $fractus$ - gebrochen) ist ein von Mathematiker Benoît Mandelbrot geprägter Begriff, der bestimmte natürliche oder künstliche Gebilde oder Muster bezeichnet. Diese Gebilde oder Muster weisen einen hohen Grad von Skaleninvarianz bzw. Selbstähnlichkeit. 
Das ist beispielsweise der Fall, wenn ein Objekt aus mehreren verkleinerten Kopien seiner selbst besteht. \cite{fraktal} \\
Figure \ref{fig:frac_kunst} stellt ein Beispiel für ein künstliches Fraktal, und Figure \ref{fig:frac_math} für ein mathematisches Fraktal.
\begin{figure}[ht]   
	\begin{subfigure}{.5\textwidth}
	\centering
	\includegraphics[width=.4\linewidth]{figures/Romanesco}
	\caption{Romanesco. Ein natürliches Fraktal. \cite{fractal_romanesco}}
	\label{fig:frac_kunst}
\end{subfigure}%
\begin{subfigure}{.5\textwidth}
	\centering
	\includegraphics[width=.4\linewidth]{figures/Mandelbrot}
	\caption{Mandelbrot. Ein künstliches Fraktal. \cite{fractal_mandelbrot}}
	\label{fig:frac_math}
\end{subfigure}%
\end{figure}
\subsection{Newton Fraktale}
Diese Fraktale erscheinen sich, wenn man das Newton Verfahren für Auffinden der Wurzeln der nichtlinearen Gleichungen auf Komplexe Ebene benutzt.
Genauer gesagt, soll man die Wurzel für jeden Punkt des gesuchten Bildes mit Newton Verfahren finden.\\
Zum Beispiel nehmen wir die Funktion $f(x) = x^3 -1$. Diese Funktion hat drei Lösungen auf Komplexe Ebene: $1$, $-\sqrt[3]{-1}$ und $(-1)^{2/3}$. Näherungswerte in Koordinatenform sind $(1, 0)$, $(-0.5, 0.866)$ und $(-0.5, -0.866)$. Die Punkte des Bildes, die sich durch das Newton Verfahren zu entsprechenden Wurzeln annähern, werden entsprechend mit rot, blau und grün gefärbt.
Das Fraktal auf dem Bild \ref{fig:output3_0} repräsentiert die gewählte Umgebung.
Die Ursachen zu dieser komischen Mischung werden in der Sektion \ref{sec:analy} erläutert.
\begin{figure}[ht]   
		\centering
		\includegraphics[width=.6\linewidth]{figures/output3_0}
		\caption{Newton Fraktal für $f(x)=x^3-1$ }
		\label{fig:output3_0}
\end{figure}
\section{Visualisierung}\label{sec:vis}
In diesem Abschnitt wird ein Programm entwickelt, das Bilder der Newton Fraktale generiert.
Außerdem es wird ein Beispiel vorgestellt, wie man Animationen erzeugen kann.
\subsection{Bildgenerierung}\label{subs:vis:bild}
Als Programmiersprache wurde Java gewählt.

\if 0

\fi
\subsection{Animation}\label{subs:vis:anime}
\if 0

\fi
\section{Analyse}\label{sec:analy}
\if 0

\fi
\section{Zusammenfassung}
\if 0

\fi
% Literaturverzeichnis ------------------------------------------------
\newpage
\bibliographystyle{alphadinLinkLocal}
\bibliography{literatur} 

%\iffalse
\end{document}
%\fi
